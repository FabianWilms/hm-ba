% ----------------------------------------------------------------------------------------------------------
% Verzeichnisse
% ----------------------------------------------------------------------------------------------------------
% TODO Typ vor Nummer
\renewcommand{\cfttabpresnum}{Tab. }
\renewcommand{\cftfigpresnum}{Abb. }
\settowidth{\cfttabnumwidth}{Abb. 10\quad}
\settowidth{\cftfignumwidth}{Abb. 10\quad}

\titlespacing{\section}{0pt}{12pt plus 4pt minus 2pt}{2pt plus 2pt minus 2pt}
\singlespacing
\rhead{INHALTSVERZEICHNIS}
\renewcommand{\contentsname}{II Inhaltsverzeichnis}
\phantomsection
\addcontentsline{toc}{section}{\texorpdfstring{II \hspace{0.35em}Inhaltsverzeichnis}{Inhaltsverzeichnis}}
\addtocounter{section}{1}
\tableofcontents
%\pagebreak
\rhead{VERZEICHNISSE}
\listoffigures
%\pagebreak
\listoftables
%\pagebreak
\renewcommand{\lstlistlistingname}{Listing-Verzeichnis}
{\labelsep2cm\lstlistoflistings}
%\pagebreak

% ----------------------------------------------------------------------------------------------------------
% Abkürzungen
% ----------------------------------------------------------------------------------------------------------
\section{Abkürzungsverzeichnis}
\begin{acronym}[HATEOAS] % längste Abkürzung steht in eckigen Klammern
	\acro{HATEOAS}{Hypermedia as the engine of application state}
	\acro{LDAP}{Lightweight Directory Access Protocol}
	\acro{REST}{Representational State Transfer}
	\acro{CORS}{Cross-Origin Resource Sharing}
	\acro{CDCT}{Consumer Driven Contract Testing}
	\acro{DDD}{Domain-Driven Design}
	\acro{LHM}{Landeshauptstadt München}
	\acro{RPC}{Remote Procedure Call}
	\acro{API}{Application Programming Interface}
	\acro{ORM}{Object-Relational Mapping}
	\acro{IaC}{Infrastructure as Code}
	\acro{GUI}{Graphical User Interface}
	\acro{DNS}{Domain Name System}
	\acro{URI}{Uniform Resource Identifier}
	\acro{JPA}{Java Persistence API}
	\acro{VCS}{Version Control System}
	\acro{CI}{Continous Integration}
	\acro{VM}{Virtual Machine}
\end{acronym}
\newpage